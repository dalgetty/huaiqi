\chapter{Sensory feedback system: design and testing}
\label{chap_sensory_feedback_system}

%\section{WiseSkin system}
%The absence of intuitive tactile feedback in commercial robotic hand prostheses has negative impacts on user experience and on the adoption of the technology. The situation of hand amputees is particular to each case, it is hard to design efficient one-fits-all tactile feedback solutions. We present in this paper a tactile sensory feedback system for hand amputees with referred hand phantom sensation. The tactile feedback system consists of wireless tactile sensors, a flexible power distribution system inside soft skins, a data processing unit, and vibrotactile stimulation array. The whole sensory tactile feedback system has been integrated with a hand prosthesis. Pilot experiments have been conducted with an amputee subject in both passive multi-site tactile stimuli discrimination tasks and active object manipulation tasks. Experimental results demonstrate that the tactile sensory feedback system is intuitive to use and has positive impacts on the functionality of the hand prosthesis.
%
%According to the Amputee Coalition, there are an estimated 500,000 upper limb amputees worldwide using limb prosthetic products (Amputee Coalition). The major causes of amputation are disease, for example, diabetes, and traumatic events. There are three main types of hand prosthesis on the market: cosmetic, body-powered, and myoelectric. Cosmetic prostheses only mimic the appearance of the hand, they provide no motor or sensory functions. Body powered prostheses are usually actuated via cable systems. The system is fixed to the shoulder or harnessed to the healthy wrist. Myoelectric prostheses are the most advanced, and are controlled by tiny electrical signals generated by muscle contraction. Myoelectric prostheses have the potential to provide the most dexterous control (BeBionic),(Touch Bionics: I-Limb...), but they face more than 50\% rejection rate. One of the reasons is that commercial myoelectric prostheses offer no tactile sensory feedback to hand amputees. Providing sensory feedback is important, especially for active prosthesis users (Biddiss 2007). The sensory feedback is needed in a prosthesis to enhance the feeling of body-ownership (Marasco 2011), (Johansson) and increase object manipulation capacities (Saunders 2011). Meanwhile, it has the potential to reduce phantom limb pain.
%
%There are invasive and non-invasive approaches to providing hand amputees with tactile sensory feedback. The invasive approaches stimulate either related sensory regions in the central nervous system with cortical electrodes (Tabot 2015), or the peripheral nervous system with cuff electrodes (Ortiz-Catalan 2014),(Tan 2014), or transversal intrafascicular multichannel electrodes (Raspopovic 2014). The invasive feedback techniques provide rich information but still have a long way to go before widespread clinical application, because of concerns in infection, among other issues. Non-invasive tactile feedback systems apply various types of stimuli on the skin surface. The stimuli can be electrical current (electrotactile), vibration (vibrotactile) or mechanical stimulation (mechanotactile) on the skin to transfer tactile information. Electrotactile feedback is energy efficient but has the drawback of easily invoking uncomfortable or even painful feelings and changing skin conductivity, affecting the EMG control signal. Among the other two methods, vibrotactile stimulation devices generally have relatively a faster response, lower power consumption, and compact sizes. Therefore, it is chosen as the tactile feedback device in this study.
%
%Previous research has demonstrated the effectiveness of close-loop myoelectric prostheses (myoelectric prosthesis integrated with sensory feedback) (Antfolk 2012, Patterson 1992, Rosenbaum-Chou 2016).
%
%Although it is challenging to provide a general tactile feedback solution for hand prosthesis users, individual solutions for target groups are feasible. This study designed and evaluated a tactile feedback solution to restore the sense of touch for upper limb amputees who have evolved referred hand phantom sensation after hand amputation. Referred hand phantom sensation is the phenomenon of stimulating certain skin surface areas (usually on the hand stump or on the face) is perceived vividly and intuitively by hand amputees as touching the missing fingers. Because of these characteristics, it is feasible to apply tactile feedback on the referred hand phantom sensation and display efficient sensory tactile information from hand prostheses.
%
%This paper ?? which paper? ?? described the sensory tactile feedback system, as well as pilot studies with a hand amputee. The remainder of the paper is structured as follows: in the subsequent section, the close-loop sensory feedback system is described. Then, in Section 3 , the experimental setups and experimental procedure are introduced. The experimental results are represented and discussed in Section ???. Finally, the study was summarized and its implication and some open research questions are examined.
%
%
%We integrated a commercial hand prosthesis (Ottobock Digital Twin 8E38=7 \cite{ottobock_digital_twin}) with wireless sensory skins on the fingers and a tactile feedback device inside the socket. The modified hand prosthesis is evaluated with hand amputees who have developed referred hand phantom sensation after amputation. The system setup is illustrated in Fig. \ref{fig:wiseskin_chapter5}.
%
%
%\begin{figure}[htb!]
%    \centering
%        \includegraphics[width=0.8\textwidth]{images/WiseSkinChapter5.png}
%        \caption{The system illustration of the WiseSkin sensory feedback.}
%        \label{fig:wiseskin_chapter5}
%\end{figure} 
%
%The experimental device tested with amputee subjects consists of a clinically certified hand prosthesis with tactile sensors integrated on the fingers and a tactile feedback device inside the prosthesis socket placed on the referred hand phantom. The tactile feedback device is controlled by a micro-controller and is powered by a separate battery pack. The hand prosthesis used in the study is Ottobock Digital Twin 8E38=7, which has a large user base. An amputee wears the modified hand prosthesis with tactile feedback to perform a predefined set of activities, for example, finger and force level detection. The integrated hand prosthesis is shown in Figure \ref{fig:wiseskin_prosthesis}.
%
%\begin{figure}[htb!]
%    \centering
%        \includegraphics[width=0.8\textwidth]{images/WiseSkinProsthesis.png}
%        \caption{A commercial hand prosthesis integrated with sensory skin and a tactile feedback device}
%        \label{fig:wiseskin_prosthesis}
%\end{figure} 
%
%
%
%\subsection{Wireless tactile skin}
%Tactile sensor review, modeling, design and testing
%
%\begin{figure}[htb!]
%    \centering
%        \includegraphics[width=0.8\textwidth]{images/SensoryNode.png}
%        \caption{The wireless sensory node.}
%        \label{fig:wiseskin_chapter5}
%\end{figure} 
% 
%\textcolor{red}{rewrite the following section}In WiseSkin, the skin system fulfills three roles:
%a) to power each sensor nodes, b) to act as a waveguide for wireless communication, and c) to maintain electromechanical integrity when the finger bends
%The design and principle of the power distribution system is presented Figure 4. The skin system is designed to fit the palm of the hand prosthesis. A 3D-printed elastomer scaffold (TangoBlack, Stratasys) was structured with holes matching the shape of the sensors and the relay antenna. Once the the sensor nodes and relay antenna were inserted into the scaffold, two metallized planes were placed on the top and bottom of the scaffold. The metallized planes and sensor nodes were electrically and mechanically connected with conductive adhesive (Creative Materials). 
%We also fabricated fully elastomeric skin systems fitting one finger, as shown Figure 4c. In this case the scaffold was molded in polydimethylsiloxane (PDMS) silicone (Sylgard 184, Dow Corning) and the silicone planes were metallized with a stretchable biphasic (solid-liquid) thin film 1D grid pattern (Hirsch 2016). 
%3D printed plastic domes (6 mm in diameter, 2.5 mm in height) were glued on top of the pressure sensors to to ensure a single point of contact between sensor node and external applied force (see Fig. 5). The domes also enhanced the transduction of non-normal forces. 
%After fabrication of the flexible or stretchable skin systems, the top metallized plane was connected to a 3V DC voltage source while the bottom plane was grounded.
% 
%\begin{figure}[htb!]
%    \centering
%        \includegraphics[width=0.8\textwidth]{images/PowerDistributionSystem.png}
%        \caption{Power distribution system. a) General view of the power distribution system. Wireless sensor nodes are embedded in a silicone scaffold and sandwiched between two metallized planes that power the sensor nodes and form a wave-guide for wireless communication. b) Wireless sensor nodes and relay antenna inserted into a 3D printed silicone scaffold. Scale bar is 20 mm. c) Wireless sensor node inserted in a molded silicone scaffold and sandwiched between two metallized PDMS planes. Scale bars are 20 mm.}
%        \label{fig:power_distribution_system}
%\end{figure} 
%
%
%\subsection{Characterization of the wireless pressure sensors embedded in the skin system}
%The skin system was attached to the Ottobock hand prosthesis using tape. We applied a controlled force on the sensors embedded in the skin system using a load frame (Criterion C42, MTS) equipped with a 100 N load cell (MTS). The hand was tied to a stand and positioned with its palm facing the load frame indenter. Pressure data from the sensor nodes was streamed to an iPad via Bluetooth Low Energy (BLE) protocol. We developed a custom LabView interface to visualize and save data from the iPad on a PC. Acquisition rates were 10 Hz for load cell (force) data and 16.6 Hz for the sensor nodes (pressure) data. The indentation speed was kept constant to 0.2 mm/s while the maximum applied force was varied between 0.5 N and 25 N. 
%As shown Figure 5 sensors displayed low hysteresis and large signal to noise ratio (RMS noise of the ST LPS25HTRsensor is 1 Pa) in the 0 N to 25 N force range. Over the five sensor nodes characterized, two displayed a larger sensitivity and lower linearity. However, with the exception of sensor S2, all sensors had a repeatable response in the 0 to 25 N force range. Sensors S2 to S5 were embedded in the flexible, palm-sized skin system while sensor S6 was embedded in a fully elastomeric skin system.
%
%\begin{figure}[htb!]
%    \centering
%        \includegraphics[width=0.8\textwidth]{images/SensorCharacterization.png}
%        \caption{Charaterization of the sensor nodes inserted in the power distribution system and mounted on the Ottobock prosthesis. a) A 3D printed dome is placed above the sensor node to limit the interaction to a single point. A controlled force is applied using a load frame. Scale bars are 5 mm. b) Output of a sensor as a function of applied force. 10 loading and unloading cycles are represented. c) Output of five sensors as a function of applied force. Error bars represent standard deviation (n=10).}
%        \label{fig:sensor_characterization}
%\end{figure} 
%
%
%\subsection{Tactile feedback array}
%There are two aspects to consider when providing sensory feedback: localization (the ability to locate where the stimulation is) and intensity identification (how many stimulation levels a person can distinguish). To achieve intuitive vibrotactile stimulation, the desirable vibration frequency is about 250Hz to stimulate the Pacinian corpuscles. The device should also be compact to integrate with a hand prosthesis. Two types of pancake vibrators are proper: linear resonant actuator (LRA) and eccentric rotating mass (ERM). Previous experimental results suggest that ERM provide better intensity identification performance (Huang 2016) and are used in this study. The device is commonly used in electric toothbrushes and vibrating touch interfaces. The robustness and safeness has been proven in these applications.
%
%
%
%
%\subsection{System characterization}
%Table \ref{tab:wiseskin_characterization} lists the characterized parameters of integrated sensory feedback system. 
%
%%\begin{table}[h!]
%%\centering
%%\caption{Sen}
%%\begin{tabular}{| >{\centering\arraybackslash}m{2cm} | >{\centering\arraybackslash}m{5cm} | >{\centering\arraybackslash}m{5cm} | >{\centering\arraybackslash}m{3cm} |}
%%  \hline
%%Weight &  g\\ 
%%  \hline
%%Communication speed    &  \\  
%%\hline
%%Stimulation modality    &  Vibrotactile \\
%%\hline
%%    &  \\
%%\hline
%%BT        & 2$log_2k$  & 5 \\
%%\hline
%%\end{tabular}
%%\label{tab:wiseskin_characterization}   
%%\end{table}
%
%
%
%
%\section{Solenoid sensory feedback for gripper prosthesis}
%Our preliminary experimental results have indicated that multi-site stimulation can be disturbing as well as redundant for most of current prosthesis, which have only one degree-of-freedom. 
%
%In this system, one solenoid actuator representing the grasping force is embedded into the socket. The vibration frequency is proportional to the grasping force. 
%
%\subsection{System integration}
%The sensory feedback include force sensing resistor, communication protocol, and solenoid actuators. 
%
%The force sensing resistors are incapsulated in a custom designed fingertip compartment. 
%
%\subsection{Sensory signal processing}
%Kalman filter
%
%\subsection{Subjects}
%The testing subject was a unilateral transradial amputee, who has been using his prosthesis for more than 30 years. The testing subject lost his left (non-dominant) arm due to a traumatic event.
%
%\subsection{Experimental procedure}
%
%
%
%
%\subsubsection{Passive tactile experiments}
%
%
%
%
%\subsubsection{Deformability test}
%The deformability testing consists of two experiments: the first one is the absolute deformability identification and the second one is comparative deformability identification. During the first set of experiemnts, the subjects were given five sponges with different deformability (Fig. \ref{fig:sponges}), and were required to rank them from the softest to the hardest. The second set of experiments require the testing subject to identify the harder objects when given two objects sequentially. 
%
%\begin{figure}[htb!]
%    \centering
%        \includegraphics[width=0.8\textwidth]{images/Sponges.png}
%        \caption{Five equisized sponges with different deformability.}
%        \label{fig:sponges}
%\end{figure} 
%
%The participant has 5 minutes of training. He will then be given 3 sets of testing objects. Each set consists of 10 objects. The participant has to rate the deformability between 1 and 5 with and without sensory feedback. 5 is the hardest (the wood) and 1 is the softest (the white sponge).
%
%\subsubsection{Moving heavy and fragile objects}
%
%%\begin{figure}[htb!]
%%    \centering
%%        \includegraphics[width=0.8\textwidth]{images/Handel_Heavy_Fragile_Objects.jpg}
%%        \caption{Five equisized sponges with different deformability.}
%%        \label{fig:sponges}
%%\end{figure}
%%Handel_Heavy_Fragile_Objects
%
%
%
%
%
%
%
%
%
%
%
%\subsection{Experimental results}
%\subsubsection{Passive experimental results}
%
%
%
%
%
%\subsubsection{Deformability test}
%
%
%
%
%
%\section{Discussion}
%\subsection{Object manipulation ability}
%
%
%
%\subsection{Interview with amputees}
%
%
%
%
%
